\documentclass[a4paper, 10pt]{article}

\usepackage[utf8]{inputenc}
\usepackage[T1]{fontenc}
\usepackage[french]{babel}
\usepackage{graphicx}

\author{Jean-Michel Pasletemps}
\date {18 février 2025}
\title{Troisième Exercice}

\begin{document}

\tableofcontents
\maketitle

\section{L'histoire d'un Tux}

Il était une fois un petit pingouin appelé Tux.  Il était heureux et en bonne santé, mais il ne ressemblait à aucun autre pingouin comme vous pouvez le constater :

\begin{figure}[!ht]
\centering
\includegraphics[scale=0.7]{tux.jpeg}
\caption{Tux en vacance}
\label{fig:tux}
\end{figure}

Comme vous l'avez vu dans la figure~\ref{fig:tux} ce petit pingouin aime se dorer la pillule au soleil ; avec un petit cocktail à la main. 

\section{Mon beau tableau}

Voilà un joli tableau listant les différents Personnages de Blabla :

\begin{table}[!ht]
\centering
\begin{tabular}{|c||c|}
\hline
Nom & Rôle \\
\hline
\hline
Blabla & personnage principal \\
\hline
Wilbur Disquedur & père de Blabla \\
\hline
Clic la Souris & meilleure amie de Blabla\\
\hline
\end{tabular}
\caption{Liste non-exhaustive des personnages de l'émission}
\label{tab:Blabla}
\end{table}

Cette magnifique émission, avec de magnifiques personnages (voir tableau~\ref{tab:Blabla}), qui est malheureusement terminée, à rythmé l'enfance de beaucoup d'entre nous !

\end{document}