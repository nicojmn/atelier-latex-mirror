\documentclass[a4paper,12pt]{scrartcl}
\usepackage[utf8]{inputenc}
\usepackage[T1]{fontenc}
\usepackage[french]{babel}

\subject{LMIAM1230}
\title{Synthèse gastronomique de deux pays}
\author{Adrien \and Nicolas}

\begin{document}

\maketitle

\section{De la gastronomie belge}
	\subsection{Résumé}
	La gastronomie belge est réputée pour ses frites croustillantes, ses chocolats artisanaux de qualité supérieure,
	ainsi que ses bières variées et savoureuses.


	Les sauces populaires pour accompagner les frites incluent la mayonnaise, la sauce andalouse, la sauce samouraï et
	la sauce béarnaise.


\section{De la gastronomie népalaise}
	\subsection{Influences et plats typiques}
	La gastronomie népalaise est influencée par la cuisine indienne et tibétaine, avec des plats comme le dal bhat 
	(riz et lentilles), le momo (boulettes de pâte farcies) et le thukpa (soupe de nouilles).

	\subsection{Desserts}
	Voici quelques desserts traditionnels népalais :

		\begin{itemize}
			\item Khajuri : un biscuit sucré aux dattes
			\item Juju Dhau : un yaourt sucré, souvent servi lors de célébrations et de festivals.
		\end{itemize}

\end{document}
