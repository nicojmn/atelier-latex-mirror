\documentclass[a4paper,12pt]{scrartcl}
\usepackage[utf8]{inputenc}
\usepackage[T1]{fontenc}
\usepackage[french]{babel}

\subject{LJOKE1230}
\title{Synthèse du cours de Calembours I}
\author{Adrien \and Louis}

\begin{document}

\maketitle

\section{Analyse}
\subsection{Fondements}
Les démonstrations à connaître sont: implication, contraposition, equivalence et récurrence. 

Les relations possibles sont: réflexive, symétrique, transitive ou antisymétrique.

\section{Maths discrètes}
\subsection{Définitions}
Quel est le comble pour un cosinus ? Attraper une sinusite !

\subsection{Principe des tiroirs}
Logarithme et exponentielle sont dans un bateau. Tout à coup, Logarithme s'exclame, paniquée : Attention, on dérive !. Exponentielle lui répond : Je m'en fiche !

\begin{itemize}
	\item Le Louvain-li-Nux n'est pas responsable de la qualité de ces blagues.
	\item Ce sont des blagues dignes d'un mécatro...
\end{itemize}

\end{document}

