% document de type rapport
\documentclass[a4paper,12pt]{report}

% d�claration des variables utilis�e par le style rapport.sty
\def\hautgauche {UCL - ELI - MIAE}
\def\hautmilieu {Thesis project}
\def\hautdroite {September 2010}

% importation des packages
\usepackage{rapport}
\usepackage[latin1]{inputenc}
%\usepackage{here}
\usepackage{floatflt}
\usepackage{moreverb}
\usepackage{graphicx}
\usepackage{textcomp}
\usepackage{amsmath,amsfonts,amssymb}
\usepackage{graphicx}
\usepackage{colortbl}
%\usepackage{hyperref}
\usepackage{color}
\usepackage{verbatim}
\usepackage{float}
\usepackage{fancybox}
\usepackage{lastpage}
\usepackage{multirow}
\usepackage{textcomp} %degree
\usepackage{tabularx}
%\floatplacement{figure}{H}
% d�finition des couleurs utilis�es dans le document
% pour les liens et les diff�rents tags, commandes, ...
\definecolor{webblue}{rgb}{0,0,1}
\definecolor{webdarkblue}{rgb}{0,0,0.5}
\definecolor{webgreen}{rgb}{0,0.5,0}
\definecolor{webred}{rgb}{0.5,0,0}

% auteur et titre "fictifs"
\author{Thomas Vanzieleghem}
\title{fichier de base Latex}

% d�but du document en lui-m�me
\begin{document}

% on appelle les parties "sections"
%\def\chaptername{}

%PAGE DE GARDE---------------------------------------------------------------
\thispagestyle{empty}


\begin{table}[H]
\begin{center}
\begin {tabular}{|l||c|}
\hline
\textit{Inventaire} & \textbf{Nombre} \\
\hline
Chemises  & 4   \\
Pulls     & 12  \\
Pantalons & 1   \\
\hline
\end{tabular}
\caption{Tableau relatif � l'inventaire}
\end{center}
\end{table}

Sur la figure~\ref{ucl}, vous pouvez voir le logo UCL mis � 50 \% de la largueur du texte\\

\begin{figure}[H]
	\centering
		\includegraphics[width=0.50\textwidth]{logo-ucl.jpg}
	\caption{Voici le logo UCL}
	\label{ucl}
\end{figure}

Voici l'environnement \textbf{itemize} qui vous permet de cr�er des listes\\

\begin{itemize}
\item Premier �l�ment de la liste
\item Deuxi�me �l�ment de la liste
\item Troisi�me �l�ment de la liste
\item Quatri�me �l�ment de la liste
\item ...
\item Ni�me �l�ment de la liste
\end{itemize}
\vspace{1cm}

On peut ajouter une formule math�matique dans du texte entre deux symboles \textbf{\$}\\
\begin{center}
$\sin(x)$
\end{center}


Un environnement �quation est pr�vu pour des formules plus longues, elles seront automatiquement centr�es et num�rot�es pour �tre r�f�renc�es (�quation~\ref{eq:pnorm})\\
\begin{equation} \label{eq:pnorm}
p(x)=\frac{1}{\sigma \sqrt{2\pi}} \exp \left(-\frac{(x-\mu)^2}{2\sigma^2}\right)
\end{equation}

\pagebreak
%FIN PAGE DE GARDE-----------------------------------------------------------

% table des mati�res
%\tableofcontents

% fin du document
\end{document} 