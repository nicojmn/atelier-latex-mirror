\documentclass[10pt,svgnames,usenames,table]{beamer} %,handout si version papier
\NeedsTeXFormat{LaTeX2e}

\usepackage[T1]{fontenc}
%\usepackage[utf8x]{inputenc}
\usepackage[french]{babel}
\usepackage{lmodern}
\usepackage{amsmath,amsthm,amssymb}        % un packages mathématiques
\usepackage{pifont}
\newcommand{\cmark}{\ding{51}}%
\newcommand{\xmark}{\ding{55}}%
\usepackage{xcolor}         % pour définir plus de couleurs
\usepackage{graphicx}       % pour insérer des figures
\usepackage{lmodern}
\usepackage{url}
  \urlstyle{sf}
\usepackage{lastpage}
\usepackage{endnotes}
\usepackage{listings}
%\usepackage{listingsutf8}

\usepackage{siunitx}
\usepackage{circuitikz}
\usepackage{chemfig}
\usepackage[version=3]{mhchem}

\usepackage{wrapfig}
\usepackage{pdfpages}
\usepackage{verbatim}
\usepackage{graphicx}

\usepackage{xspace}
\usepackage{mdframed}
\usepackage{pgfplots}
\usepackage[french]{varioref} % \vpageref
\usepackage{epstopdf}
\usepackage{pdflscape} %% portrait

\usepackage{newunicodechar}

\usepackage[outputdir=../build_latex/, cachedir=../build_latex/, cache=true]{minted} % Beautiful code display

\newcommand{\badet}{et}
\newcommand{\goodet}{\mathbin{\mathrm{et}}}

\DeclareMathOperator{\sumN}{\sum_{i=1}^n}
\DeclareMathOperator{\var}{\mathrm{Var}}

% THEME
% voir (http://mcclinews.free.fr/latex/beamergalerie.php)
%\usetheme{Singapore} % Thème général; les lignes suivantes le recréent en le personnalisant un peu
\setbeamercolor{section in head/foot}{use=structure,bg=structure.fg!25!bg} % "Amélioration du jeu de couleur"
\useoutertheme[subsection=false,footline=institutetitle]{miniframes} % Gère les bullets de navigation en topbar. Options : pas de rappel du titre de ss-section + un footer
\setbeamerfont{frametitle}{series=\bfseries}
\setbeamertemplate{frametitle}[default][center] % Titre centré et bien placé.

\usefonttheme[onlymath]{serif} % to see the difference when I do mathsf
%\renewcommand\sfdefault{cmss} % Polices

% COULEURS PERSO
\definecolor{gris}{RGB}{228,228,228}
\definecolor{bleu}{RGB}{34,148,255}
\definecolor{darkgray}{rgb}{0.3,0.3,0.3}
\definecolor{codeGreen}{rgb}{0,0.5,0}

% OPTIONS POUR LES LSTLISTINGS IMBRIQUES
\lstset{
      language=TeX,
      flexiblecolumns=true,
      numbers=left,
      stepnumber=1,
      numberstyle=\ttfamily\tiny,
      keywordstyle=\ttfamily\textcolor{blue},
      stringstyle=\ttfamily\textcolor{red},
      commentstyle=\ttfamily\textcolor{codeGreen},
      breaklines=true,
      extendedchars=true,
      basicstyle=\ttfamily\scriptsize,
      showstringspaces=false,
      morekeywords={usepackage,documentclass,begin,textbf,textit,texttt,ref,includegraphics,caption,label,setlength,mathbb,notag,frac,num,si,ang,SI,textwidth,percent,meter,ohm,joule,second,more,section,subsection,tableofcontents,setstretch,TeX,LaTeX,sffamily,emph,chemfig,pageref,vpageref,date,maketitle,institute,author,and,textsc,title,includeonly,include,clearpage,newcommand,mathsf,renewcommand,per,celsius,square,volt,cubic,lumen,farad,DeclareMathOperator,mathrm,mathcal,captionof,lstinputlisting,lstinline,tiny,small,normalsize,large,Large,huge,Huge},
      frame=single,
      extendedchars=true,
      inputencoding=utf8x,
	    literate={á}{{\'a}}1 {ã}{{\~a}}1 {é}{{\'e}}1 {è}{{\`e}}1 {à}{{\`a }}1
    }
%\lstset{inputencoding=utf8/latin1}
\lstset{literate=
{á}{{\'a}}1
{à}{{\`a}}1
{À}{{\`A}}1
{ã}{{\~a}}1
{é}{{\'e}}1
{è}{{\`e}}1
{ê}{{\^e}}1
{î}{{\^i}}1
{í}{{\'i}}1
{ó}{{\'o}}1
{õ}{{\~o}}1
{ô}{{\^o}}1
{ú}{{\'u}}1
{ü}{{\"u}}1
{ç}{{\c{c}}}1
}
\lstset{keepspaces}
\lstdefinestyle{nonumbers}
{numbers=none}

% NAVIGATION SYMBOLS
% Pour personnaliser la barre de navigation du dessous
\setbeamertemplate{navigation symbols}{
	%\insertslidenavigationsymbol
	%\insertframenavigationsymbol
	%\insertsubsectionnavigationsymbol
	\quad\textbf{\insertframenumber/\inserttotalframenumber} % Numéro de page
	%\insertsectionnavigationsymbol
	%\insertdocnavigationsymbol
	%\insertbackfindforwardnavigationsymbol
}
% Supprimer les icones de navigation (pour les transparents)
%\setbeamertemplate{navigation symbols}{}

% Mettre les icones de navigation en mode vertical (pour projection)
% \setbeamertemplate{navigation symbols}[vertical]


% CUSTOM DES ITEMIZE
\setbeamertemplate{itemize item}[ball]
\setbeamertemplate{itemize subitem}[triangle]
\setbeamertemplate{itemize subsubitem}[circle]

% "Fioriture de style" : qd <x-> dans les item, les autres en gris clair
\beamertemplatetransparentcovered

% Les block arrondis et ombrés dans la couleur que je veux
\setbeamertemplate{blocks}[rounded][shadow=true]
\definecolor{normalBlockColor}{RGB}{255,255,255}
\definecolor{normalTitleBlockColor}{RGB}{0,0,102}
\definecolor{normalBlockTextColor}{RGB}{0,0,0}
\definecolor{normalBlockTitleTextColor}{RGB}{255,255,255}
\definecolor{exampleBlockColor}{RGB}{202,251,197}
\definecolor{exampleTitleBlockColor}{RGB}{166,241,158}
\definecolor{exampleBlockTextColor}{RGB}{0,0,0}
\definecolor{exampleBlockTitleTextColor}{RGB}{0,120,0}
\definecolor{alertBlockColor}{RGB}{248,218,218}
\definecolor{alertTitleBlockColor}{RGB}{244,108,108}
\definecolor{alertBlockTextColor}{RGB}{0,0,0}
\definecolor{alertBlockTitleTextColor}{RGB}{120,0,0}
\setbeamercolor*{block title}{fg=normalBlockTitleTextColor,bg=normalTitleBlockColor}
\setbeamercolor*{block body}{fg=normalBlockTextColor,bg=normalBlockColor}
\setbeamercolor*{block title alerted}{fg=alertBlockTitleTextColor,bg=alertTitleBlockColor}
\setbeamercolor*{block body alerted}{fg=alertBlockTextColor,bg=alertBlockColor}
\setbeamercolor*{block title example}{fg=exampleBlockTitleTextColor,bg=exampleTitleBlockColor}
\setbeamercolor*{block body example}{fg=exampleBlockTextColor,bg=exampleBlockColor}
\setbeamerfont{block title}{size={}}

% TABLES OF CONTENT
% Pour rendre les toc plus compactes (pour éviter que ça déborde)
\makeatletter
\patchcmd{\beamer@sectionintoc}{\vskip1.5em}{\vskip0.5em}{}{}
\makeatother
\setbeamerfont{subsection in toc}{size=\scriptsize}


\graphicspath{{img/}}


\newcommand\Warning{%
 \makebox[1.4em][c]{%
 \makebox[0pt][c]{\raisebox{.1em}{\small!}}%
 \makebox[0pt][c]{\color{red}\Large$\bigtriangleup$}}}%

\newunicodechar{⚠}{\Warning}


\usepackage{pdflscape} %% portrait
\usepackage[french]{varioref} % \vpageref
\usepackage{pgfplots}

\newcommand{\badet}{et}
\newcommand{\goodet}{\mathbin{\mathrm{et}}}

\graphicspath{{img/}}
\definecolor{gris}{RGB}{228,228,228}
\definecolor{bleu}{RGB}{34,148,255}
\definecolor{darkgray}{rgb}{0.3,0.3,0.3}

\usefonttheme[onlymath]{serif} % to see the difference when I do mathsf

\logo{\includegraphics[height=5mm]{logo_12-13-mini.png}}
\institute{Louvain-li-Nux}
\title{\textbf{Formation \LaTeX}\\
Introduction à l'écriture de document \textrm{\LaTeX}}
\author{Xavier \textsc{Lambein} \and Geoffroy \textsc{Jacquet}}

%\date{24 mars 2015}


% A changer (selon Arnaud) le utf8x pas aligné slide 12/30
% maketitle
%
%
\begin{document}

%%%%%%%%%%%% SIDA
\begin{landscape}
\begin{frame}[noframenumbering,plain]
	\vspace{-.5cm}
	\hspace*{.1mm}
	\includegraphics[page=1,height=\paperwidth]{latex_sida.pdf}
\end{frame}
\end{landscape}
%%%%%%%%%%%%

\begin{frame}
\maketitle
Merci à Jolan \textsc{Wolter} et Thomas \textsc{Vanzieleghem} pour avoir réalisé la première version de ces slides
ainsi qu'à David \textsc{Ernst} et Matthieu \textsc{Baerts} pour avoir réalisé la deuxième version, ainsi qu'à Arnaud \textsc{Cerckel} et Benoît \textsc{Legat} pour avoir réalisé la troisième version.
\end{frame}

\AtBeginSection[]
  {
     \begin{frame}<beamer>
     \frametitle{\insertsection}
     \tableofcontents[hideothersubsections]
     \end{frame}
  }

\section{Introduction}
\subsection{Qu'est-ce que \LaTeX{}?}
\begin{frame}
\frametitle{Qu'est ce que \LaTeX}

\begin{itemize}
\item \TeX{} $ \Rightarrow$ programme de mise en page
\vspace{0.5cm}
\item \LaTeX{} $ \Rightarrow$ ensemble de commandes qui seront
 interprétées par le programme \TeX
 \vspace{0.5cm}
\item \LaTeX{} $ \neq$ WYSIWYG (What You See Is What You Get)
\end{itemize}

\end{frame}

\subsection{Pourquoi \LaTeX{}?}
\begin{frame}{Pourquoi \LaTeX{}?}

  \begin{itemize}
  	\item Qualité professionnelle de document
	\item Facilité d'emploi des :
	\begin{itemize}
		\item formules mathématiques
		\item table des matières
		\item références bibliographiques
		\item références croisées
		\item ...
	\end{itemize}
	\item Séparation entre contenu et forme
	\item Description du contenu indépendant de la forme
	\item Gratuit
	\item Stable, même pour les très gros documents
  \end{itemize}
\end{frame}

\begin{frame}{Pourquoi \LaTeX{}?}

\begin{figure}[htbp]
\begin{center}
\includegraphics[height=6.5cm]{latex_exemples}
\end{center}
\end{figure}
\end{frame}

%-----------------------
\subsection{Pourquoi pas \LaTeX{}?}
\begin{frame}{Pourquoi pas \LaTeX{}?}

  \begin{itemize}
  	\item Les tableaux...
	\item Prise en main plus longue que pour traitement de texte WYSIWYG
	\item Je suis allergique à toute forme de code informatique
	\item J'ai des actions Microsoft
	\item Je ne trouve pas le ``\textbackslash'' sur mon clavier
  \end{itemize}
\end{frame}

\begin{frame}{Oui mais...}
  \begin{center}
    %\resizebox{\textwidth}{!}{
    \begin{tikzpicture}
      \begin{axis}[xmin=0, xmax=4, ymin=-2, ymax=5,
          xlabel={Expérience, maitrise}, ylabel=Productivité,
        legend style={at={(0.01,0.99)}, anchor=north west}]
        \addplot[smooth, color=blue]{x-0.05};
        \addlegendentry{\LaTeX}
        \addplot[smooth, color=red,domain=0:5]{sqrt(x)};
        \addlegendentry{Word}
        %\xlabel{Productivité}
        %\ylabel{Expérience/maitrise}
      \end{axis}
      \foreach \x/\com/\deltax/\deltay/\adj in {
        {1.1}/{Maintenant}/{0}/{0.8}/below,
        {2.2}/{Après la formation}/{0}/{0.4}/below,
        %{4.6}/{À l'heure de votre mémoire}/{0}/{1.2}/below
        {4.6}/{Au bout de queques semaines}/{0}/{1.2}/below
      } {
        %\fill \coord circle (100pt) node[\adj] {$\coord$};
        \draw[dashed] (\x,0) -- (\x+\deltax,\deltay) node[above] {\com};
        %\draw (\x+\deltax,\deltay) node {\com};
      }
    \end{tikzpicture}
    %}
  \end{center}
\end{frame}

%-----------------------
\subsection{Les Outils}
\begin{frame}{Ce qu'il faut pour commencer.}

  \begin{itemize}
	  \item GNU/Linux
	  \begin{itemize}
	  	\item Distribution \LaTeX{} = \textbf{TeXLive}
		\item Éditeur de texte = \textbf{TeXMaker, LaTeXila, Kile}
	  \end{itemize}
	  \item Windows
	  \begin{itemize}
	  	\item Distribution \LaTeX{} = \textbf{MikTeX}
		\item Éditeur de texte = \textbf{TeXMaker, TeXnicCenter}
	  \end{itemize}
	  \item Mac OS
	  \begin{itemize}
	  	\item Distribution \LaTeX{} = \textbf{MacTeX}
		\item Éditeur de texte = \textbf{TeXMaker, TeXShop, iTeXMac}
	  \end{itemize}
	  \item Dans votre navigateur
	  \begin{itemize}
		\item \textbf{\url{www.overleaf.com}}
  		\item \textbf{\url{www.sharelatex.com}}
	  \end{itemize}
  \end{itemize}
\end{frame}

\section{Les concepts de base}
\subsection{Les fichiers}

\begin{frame}{Les fichiers}

	\begin{center}
		\includegraphics[height=3cm]{compilation.jpg}
	\end{center}

	\begin{itemize}
		\item Fichier source = essais\alert{\textbf{.tex}}
		\item Fichier de bibliographie = essais\alert{\textbf{.bib}}
		\item Lors de compilation $\rightarrow$ création de nombreux fichiers annexes
		\begin{itemize}
			\item style, class;
			\item structure du document;
			\item table des matières, liste des figures;
			\item liste des références;
			\item ...
		\end{itemize}
		\item Création d'un fichier essais\alert{\textbf{.pdf}}
	\end{itemize}
\end{frame}

%-----------------------
\subsection{La structure}

\begin{frame}[fragile]{Structure générale du document}
\framesubtitle{Séparation du préambule et du corps du document}
\small
\begin{tabular}{ll}
  Type de document &
  \lstinline|\documentclass[a4paper, 10pt]{article}|\\
  Utilisation de \textit{package} &
  \lstinline|\usepackage[utf8x]{inputenc} % ou [utf8]|\\
  Utilisation de \textit{package} &
  \lstinline|\usepackage[T1]{fontenc}|\\
  Utilisation de \textit{package} &
  \lstinline|\usepackage[french]{babel}|\\
  Utilisation de \textit{package} &
  \lstinline|\usepackage{lmodern}|\\
  Blanc pour la lisibilité &\\
  Début du document &
  \lstinline|\begin{document}|\\
  Corps du document &
  \lstinline|Ceci est mon premier document en \LaTeX{}|\\
  Fin du document &
  \lstinline|\end{document}|\\
\end{tabular}


\end{frame}

%-----------------------

\subsection{Les classes}

\begin{frame}[fragile]{Les principales classes de document}
\begin{tabular}{lp{8cm}}
  \textbf{article} & pour les articles de journaux scientifiques, présentations, rapports courts...\\
  \textbf{report} & pour de plus long rapports de plusieurs chapitres, petits livres, thèses, ...\\
  \textbf{book} & pour de vrais livres.\\
  \textbf{letter} & pour écrire des lettres.\\
  \textbf{beamer} & pour écrire des présentations (comme celle-ci).
\end{tabular}
\vspace{1cm}
\begin{center}
\verb|\documentclass[a4paper,10pt]{|\alert{\texttt{article}}\verb|}|\\
\end{center}
\end{frame}

\subsection{Les options}

\begin{frame}[fragile]{Les principales options de document}
\begin{tabular}{lp{8cm}}
  \textbf{10pt, 11pt, 12pt} & pour la taille de police.\\
  \textbf{a4paper, a5paper} & pour la taille de page.\\
  \textbf{onecolumn, twocolumn} & pour faire plusieurs colonnes.\\
  \textbf{landscape} & pour une mise en page paysage.\\
  \textbf{twoside} & pour des marges de livre
\end{tabular}
\vspace{1cm}
\begin{center}
\verb|\documentclass[|\alert{\texttt{a4paper,10pt}}\verb|]{article}|\\
\end{center}
\end{frame}

%TODO: abstract

\subsection{La structure} %TODO: exemple
\begin{frame}[fragile]{La structure logique du document}
	\begin{itemize}
		\item Structure logique du document uniquement
		\item \LaTeX{} se charge de la numérotation et de la mise en page\\
	\end{itemize}
	\vspace{1cm}

	\verb|\part{}|\\
	\hspace{1cm} \verb|\chapter{}| \hspace{2cm}$\Longrightarrow$ \textcolor{bleu}{uniquement \textit{book} et \textit{report}}\\
	\hspace{2cm} \verb|\section{}|\\
	\hspace{3cm} \verb|\subsection{}| \\
	\hspace{4cm} \verb|\subsubsection{}| \\
	\hspace{5cm} \verb|\paragraph{}	| \\

\end{frame}

%-----------------------
\section{Mise en page générale} %TODO: faire ca plus clairement
\subsection{Titre}
\begin{frame}[fragile]{Titre}
	\begin{itemize}
		\item Automatiquement la date d'aujourd'hui dans la bonne langue grâce à \lstinline|babel|
	\end{itemize}
	\begin{columns}
      \begin{column}{0.4\textwidth}
        \begin{lstlisting}
\institute{Louvain-li-Nux}
\title{\textbf{Formation \LaTeX}\\
Introduction \`a l'\'ecriture
de document \LaTeX}
\author{Xavier \textsc{Lambein}
   \and Geoffroy \textsc{Jacquet}}
                    % today
\date{24 mars 2015} % fixed data
\date{}             % no date
\begin{document}
\maketitle
        \end{lstlisting}
      \end{column}
      \begin{column}{0.6\textwidth}
        \maketitle
      \end{column}
	\end{columns}
\end{frame}

\subsection{Le résumé ou abstract}
\begin{frame}[fragile]{Le résumé ou abstract}
	\begin{itemize}
		\item L'environnement \texttt{abstract} permet de mettre en page un résumé au début du document.
	\end{itemize}
	\begin{columns}
      \begin{column}{0.5\textwidth}
        \begin{lstlisting}
\begin{document}
...
\begin{abstract}
	Voici un résumé succint du contenu
	de mon document.
\end{abstract}
...
\end{document}
        \end{lstlisting}
      \end{column}
      \begin{column}{0.5\textwidth}
				\begin{abstract}
					Voici un résumé succint du contenu de mon document.
				\end{abstract}
      \end{column}
	\end{columns}
\end{frame}

\subsection{La table des matières}
\begin{frame}[fragile]{Table des matières}
	\begin{itemize}
		\item Une ligne de commande suffit pour générer toute la table des matières
	\end{itemize}
	\begin{columns}
      \begin{column}{0.5\textwidth}
        \begin{lstlisting}
\begin{document}
\tableofcontents % Table des matières

\section{Introduction}
Ceci est mon premier document en \TeX{}
\section{Le vif du sujet}
Le sujet est en or mais pas le vif.
\subsection{Mais quel est le sujet ?}
\LaTeX{}, ce logiciel d'exception !
\end{document}
        \end{lstlisting}
      \end{column}
      \begin{column}{0.5\textwidth}
        \begin{flushright}
          \fbox{\includegraphics[width=0.6\textwidth]{table.png}}
        \end{flushright}
      \end{column}
	\end{columns}
\end{frame}


%-----------------------
\subsection{Les polices}
\begin{frame}[fragile]{Jouer avec les fontes}
\framesubtitle{Changer la taille de police}
\begin{columns} %TODO: tableau
  \begin{column}{0.4\textwidth}
  	\begin{lstlisting}[language=tex]
{\tiny polygenelubricants}
{\small polygenelubricants}
{\normalsize polygenelubricants}
{\large polygenelubricants}
{\Large polygenelubricants}
{\LARGE polygenelubricants}
{\huge polygenelubricants}
{\Huge polygenelubricants}
		\end{lstlisting}
  \end{column}
  \begin{column}{0.6\textwidth}
		\begin{flushright}
			\tiny{polygenelubricants} \\
			\small{polygenelubricants} \\
			\normalsize{polygenelubricants} \\
			\large{polygenelubricants} \\
			\Large{polygenelubricants} \\
			\LARGE{polygenelubricants} \\
			\huge{polygenelubricants} \\
			\Huge{polygenelubricants}
		\end{flushright}
  \end{column}
\end{columns}
\end{frame}

\begin{frame}[fragile]{Jouer avec les fontes}
\framesubtitle{Changer le type et style de police}

\begin{block}{Type de police}
\begin{columns}
  \begin{column}{0.5\textwidth}
		\begin{lstlisting}
\textrm{Serif (par défaut)}
\textsf{Sans serif}
\texttt{Machine à écrire}
		\end{lstlisting}
  \end{column}
  \begin{column}{0.4\textwidth}
		\textrm{Serif (par défaut)} \\
		\textsf{Sans serif} \\
		\texttt{Machine à écrire}
	\end{column}
\end{columns}
\end{block}

\begin{block}{Style de police}
\begin{columns}
  \begin{column}{0.5\textwidth}
		\begin{lstlisting}
\emph{Emphase}
\textbf{Gras}
\textsl{Italique}
\textsc{Petites majuscules}
		\end{lstlisting}
  \end{column}
  \begin{column}{0.4\textwidth}
		\emph{Emphase} \\
		\textbf{Gras} \\
		\textsl{Italique} \\
		\textsc{Petites majuscules}
	\end{column}
\end{columns}
\end{block}
\end{frame}


\subsection{Paragraphes}
\begin{frame}[fragile]{Les paragraphes avec \LaTeX{}}
	\begin{block}{Définition d'un paragraphe}
	  Pour créer un nouveau paragraphe, il suffit de faire deux retours à la ligne
	  \begin{lstlisting}
	  Premier paragraphe.

	  Second paragraphe.
	  \end{lstlisting}
	\end{block}
	\begin{block}{Ajouter de l'espace entre les paragraphes et changer l'indentation}
	  \begin{lstlisting}
	  \usepackage{parskip} % Ajoute de l'espace entre les paragraphes et mets l'indentation to 0
	  \setlength{\parindent}{15pt} % Remets l'indentation par default
	  \end{lstlisting}
	\end{block}
  \begin{block}{Espace interligne}
    \begin{lstlisting}
    \usepackage{setspace}
    \setstretch{1.5}
    \end{lstlisting}
  \end{block}
\end{frame}

\begin{frame}[fragile]
  \frametitle{Alignement d'un paragraphe}
  Par défaut, c'est justifié.
  \begin{lstlisting}
\begin{center} % Pour centrer
\end{center}
\begin{flushright} % Pour aligner à droite
\end{flushright}
{\centering ...}
  \end{lstlisting}
\end{frame}


%-----------------------
\subsection{Listes}
\begin{frame}[fragile]
  \frametitle{Itemize et enumerate}
  \begin{columns}
    \begin{column}{0.45\textwidth}
      \begin{block}{Code}
        \begin{lstlisting}
\begin{itemize}
  \item Un chat;
  \item une poule;
  \item un chien.
\end{itemize}
        \end{lstlisting}
      \end{block}
      \begin{block}{Rendu}
        \begin{itemize}
          \item Un chat;
          \item une poule;
          \item un chien.
        \end{itemize}
      \end{block}
    \end{column}

    \begin{column}{0.45\textwidth}
      \begin{block}{Code}
        \begin{lstlisting}
\begin{enumerate}
  \item Mettez de l'eau.
  \item Chauffer l'eau.
  \item Mettez les pasta.
\end{enumerate}
        \end{lstlisting}
      \end{block}
      \begin{block}{Rendu}
        \begin{enumerate}
          \item Mettez de l'eau.
          \item Chauffer l'eau.
          \item Mettez les pasta.
        \end{enumerate}
      \end{block}
    \end{column}
  \end{columns}
\end{frame}

\begin{frame}[fragile]
  \frametitle{Description}
  \begin{block}{Code}
    \begin{lstlisting}
\begin{description}
  \item[ODT] Open Document Text.
  \item[ODS] Open Document Spreadsheet.
  \item[ODP] Open Document Presentation.
\end{description}
    \end{lstlisting}
  \end{block}
  \begin{block}{Rendu}
    \begin{description}
      \item[ODT] Open Document Text.
      \item[ODS] Open Document Spreadsheet.
      \item[ODP] Open Document Presentation.
    \end{description}
  \end{block}
\end{frame}

\subsection{Divers}
\begin{frame}[fragile]
  \frametitle{Divers}
  \begin{block}{Guillemets français}
    Utiliser \lstinline|\og{}| et \lstinline|\fg{}| et non \lstinline|"|.

    "bad" \og{}good\fg{}.
  \end{block}

 \begin{block}{Tirets}
   \begin{tabular}{llp{0.5\textwidth}}
     \lstinline|-| & mots composés & Jean-Patrick\\
     \lstinline|--| & intervals & 1984--2015\\
     \lstinline|---| & parenthèses & le \LaTeX{} ---c'est chouette--- a été créé par Leslie Lamport\\
   \end{tabular}
 \end{block}
\end{frame}

%-----------------------
\section{Les environnements flottants}
\subsection{Les figures}
\begin{frame}[fragile,allowframebreaks]
  \frametitle{Figures}
  \begin{block}{Non-flottant}
    Référencement par ``ci-dessous'', \dots
    \begin{lstlisting}
\usepackage{graphicx}
...
\begin{center}
  \includegraphics{image.jpg}
\end{center}
    \end{lstlisting}
  \end{block}
  \begin{block}{Flottant}
    Référencement par \lstinline|voir figure~\ref{fig:graphique}|
    \begin{lstlisting}
\usepackage{graphicx}
...
\begin{figure}[!ht]
  \centering
  \includegraphics{graph.png}
  \caption{Voici un beau graphique}
  \label{fig:graphique}
\end{figure}
    \end{lstlisting}
  \end{block}
  \begin{block}{Hybride: référençable mais non-flottant}
    Référencement par \lstinline|voir figure~\ref{fig:graphique}|
    \begin{lstlisting}
\usepackage{graphicx}
\usepackage{float}
...
\begin{figure}[H]
  ...
  \label{fig:graphique}
\end{figure}
    \end{lstlisting}
    ou
    \begin{lstlisting}
\usepackage{graphicx}
\usepackage{caption}
...
\begin{center}
  ...
  \captionof{figure}{Voici un beau graphique}
  \label{fig:graphique}
\end{center}
    \end{lstlisting}
  \end{block}
  \begin{block}{Scaling}
    \begin{lstlisting}
\usepackage{graphicx}
...
\includegraphics[width=\textwidth]{image.jpg} % Largeur d'une ligne de texte
\includegraphics[height=4cm]{image.jpg} % Hauteur de 4cm
\includegraphics[scale=0.5]{image.png} % taille / 2
    \end{lstlisting}
  \end{block}

  \vspace{4.2em}

\begin{quote}
  1992: Extensive testing shows that 98.3\% of the time
  no matter which of the
  \lstinline|[h]|,
  \lstinline|[t]|,
  \lstinline|[b]|, or
  \lstinline|[p]|
  options is used,
  \LaTeX{} will put your \lstinline|table| at the end of the document.

  \vspace{1em}
  \hfill
  \em
  \begin{minipage}{8cm}
    DAVID F. GRIFFITHS and DESMOND J. HIGHAM, Great Moments in \LaTeX{} History (1997)
  \end{minipage}
\end{quote}

\end{frame}

\begin{frame}[fragile]{Exemple de figure}
	\fbox{\includegraphics[width=\textwidth]{figure.jpg}}
\end{frame}

\begin{frame}[fragile]{Exemple de figure}
%\framesubtitle{Un contenant pour des éléments de type \textit{table} et \textit{figure}}

\begin{lstlisting}
\usepackage{graphicx}
...
Sur la figure~\ref{fig:ucl}, vous pouvez voir le logo UCL
mis a \SI{50}{\percent} de la largueur du texte.

\begin{figure}[!ht]
    \centering
        \includegraphics[width=0.50\textwidth]{logo-ucl.jpg}
    \caption{Voici le logo UCL}
    \label{fig:ucl}
\end{figure}
\end{lstlisting}

\end{frame}
%	\vspace{0.5cm}
%	\hspace{1cm} \verb|\begin{|{\usebeamercolor[fg]{example text} \verb|itemize|}\verb|} |\\
%	\hspace{2cm} \verb|\item Premier élément de la liste|\\
%	\hspace{2cm} \verb|\item Deuxième élément de la liste|\\
%	\hspace{1cm} \verb|\end{|{\usebeamercolor[fg]{example text} \verb|itemize|}\verb|} |\\
%	\vspace{0.5cm}

\subsection{Les tableaux} %TODO: site web
\begin{frame}[fragile,allowframebreaks]
  \frametitle{Tableaux}
  \begin{block}{Non-flottant}
    Référencement par ``ci-dessous'', \dots
    \begin{lstlisting}
\begin{center}
  \begin{tabular}{...}
    ...
  \end{tabular}
\end{center}
    \end{lstlisting}
  \end{block}
  \begin{block}{Flottant}
    Référencement par \lstinline|voir tableau~\ref{tab:data}|
    \begin{lstlisting}
\begin{table}
  \centering
  \begin{tabular}{...}
    ...
  \end{tabular}
  \caption{Voici un beau tableau}
  \label{tab:data}
\end{table}
    \end{lstlisting}
  \end{block}
  \begin{block}{Code}
    \begin{lstlisting}
\begin{tabular}{|lcr|}
  \hline
  A & B & C\\
  \hline
  a & b & c\\
  $\alpha$ & $\beta$ & $\gamma$\\
  \hline
\end{tabular}
    \end{lstlisting}
  \end{block}
  \begin{block}{Rendu}
\begin{tabular}{|lcr|}
  \hline
  A & B & C\\
  \hline
  a & b & c\\
  $\alpha$ & $\beta$ & $\gamma$\\
  \hline
\end{tabular}
  \end{block}
\end{frame}



\begin{frame}[fragile]{Exemple de tableau}
\begin{footnotesize}
\begin{lstlisting}
\begin{table}[!ht]
    \begin{center}
            \begin {tabular}{|l||c|} %% 2 columns
            \hline
                \textit{Inventaire} & \textbf{Nombre} \\
            \hline
                Chemises  & 4   \\
                Pulls     & 12  \\
                Pantalons & 1   \\
            \hline
            \end{tabular}
        \caption{Tableau relatif a l'inventaire}
    \end{center}
\end{table}
\end{lstlisting}
\end{footnotesize}
\begin{center}
\fbox{\includegraphics[width=0.6\textwidth]{table.jpg}}
\end{center}
\end{frame}

%%%%%%%%%%%%%%%%%%%%%%%%%%%
\section{Références}

\subsection{Références des éléments du texte}
\begin{frame}[fragile]
\frametitle{\insertsubsection}
\begin{itemize}
  \item Facile de faire référence à un numéro et la page d'une section et d'un environnement (\texttt{figure}, \texttt{equation}, \texttt{table}).
  \item D'un coté une étiquette :
    \begin{itemize}
      \item \lstinline|\label{id}|.
    \end{itemize}
  \item De l'autre une référence à cette étiquette :
    \begin{itemize}
      \item \lstinline|\ref{id}|
      \item \lstinline|\pageref{id}|
      \item \lstinline|\vpageref{id}| du paquet \lstinline|varioref|
    \end{itemize}
\end{itemize}
\begin{columns}
  \begin{column}{0.5\textwidth}
    \label{id}
    Nous sommes section~\ref{id},
    page~\pageref{id},
    \vpageref{id}.
  \end{column}
  \begin{column}{0.5\textwidth}
    \begin{lstlisting}
\label{ref}
Nous sommes section~\ref{ref},
page~\pageref{ref},
\vpageref{ref}.
    \end{lstlisting}
  \end{column}
\end{columns}

\end{frame}

\subsection{Footnote}
\begin{frame}[fragile]
  \frametitle{Footnote}
  \begin{lstlisting}
    The earth\footnote{mostly harmless} was destroyed
    by Vogons\footnote{They have the worst poetry in the universe}.

    But Don't Panic\footnote{By the way, the answer is 42},
    even when you're at the restaurant at
    the end of the universe.
  \end{lstlisting}
  \begin{block}{Result}
    The earth\footnote{Mostly harmless} was destroyed
    by Vogons\footnote{They have the worst poetry in the universe}.

    But Don't Panic\footnote{By the way, the answer is 42},
    even when you're at the restaurant at
    the end of the universe.
  \end{block}
\end{frame}

\subsection{Bibliographie}
\begin{frame}
\frametitle{\insertsubsection}
Pour maintenir une bibliographie, on utilise de préférence le fichier \texttt{\textbf{.bib}}, qui contient toutes les références bibliographiques.

Pour les utiliser:
\begin{itemize}
	\item Ajouter la source dans le fichier \texttt{bib}.
	\item Inclure dans son texte la commande \texttt{cite} avec l'étiquette de la source à référencer.
	\item \LaTeX{} inclut la référence dans le texte et ajoute la source à la bibliographie.
\end{itemize}
\end{frame}

\begin{frame}[fragile,allowframebreaks]
  \frametitle{Bibliography}
  \begin{columns}
    \begin{column}{0.45\textwidth}
  \begin{block}{Citer}
    \begin{lstlisting}
\cite{goossens93}
\cite[p.~42]{goossens93}
\cite{goossens93,combefis11,...}
    \end{lstlisting}
  \end{block}
  \begin{block}{Inclure la bibliographie}
    \begin{lstlisting}
\bibliographystyle{plain}
\bibliography{biblio}
    \end{lstlisting}
  \end{block}
\end{column}
\begin{column}{0.45\textwidth}
  \centering
  \includegraphics[width=\textwidth]{img/scholar_circle.png}
\end{column}
\end{columns}

\begin{description}
  \item[bad] \lstinline|voir\cite{goossens93}|
  \item[ok] \lstinline|voir \cite{goossens93}|
  \item[ok] \lstinline|voir~\cite{goossens93}|
\end{description}

\bibliographystyle{plain}
  \begin{block}{Élément d'une bibliographie}
    À mettre dans \lstinline|biblio.bib|
    \begin{lstlisting}
@book{goossens93,
    author    = "Michel Goossens and Frank Mittelbach and Alexander Samarin",
    title     = "The LaTeX Companion",
    year      = "1993",
    publisher = "Addison-Wesley",
    address   = "Reading, Massachusetts"
}
@book{knuth1986texbook,
  title={The texbook},
  author={Knuth, Donald Ervin and Bibby, Duane},
  volume={1993},
  year={1986},
  publisher={Addison-Wesley Reading, MA, USA}
}
    \end{lstlisting}
  \end{block}
\end{frame}


\subsection{include et input}
\begin{frame}[fragile]
  \frametitle{include et input}
  \begin{columns}
    \begin{column}{0.5\textwidth}
      Simple ``copier/coller''.
      \begin{lstlisting}
\input{chap1}
\input{chap2}
\input{chap3}
\input{chap4}
      \end{lstlisting}
    \end{column}
    \begin{column}{0.5\textwidth}
      \lstinline|\include{x}| c'est comme faire
      \begin{lstlisting}
\clearpage
\input{x}
\clearpage
      \end{lstlisting}
      Il y a aussi \lstinline|includeonly| pour gagner du temps
      \begin{lstlisting}
\includeonly{chap1,chap3}
...
\include{chap1}
\include{chap2}
\include{chap3}
\include{chap4}
      \end{lstlisting}
    \end{column}
  \end{columns}
\end{frame}


%-----------------------
\section{Sciences}
\subsection{Écrire des mathématiques}
\begin{frame}[fragile]{L'environnement mathématique}
\framesubtitle{Inclure des formules dans le texte}
On peut ajouter une formule mathématique dans du texte entre deux symboles \textbf{\$}.

\vspace{1cm}
\begin{center}
    \Large
  \begin{tabular}{lll}
    \verb|$x^{2n}$| & $\rightarrow$ &  $x^{2n}$\\
    \medskip
    \verb|$\sin(x)$| & $\rightarrow$ &  $\sin(x)$\\
  \end{tabular}
\end{center}
\end{frame}
\begin{frame}[fragile]{L'environnement mathématique}
\framesubtitle{Inclure des formules centrées hors du texte}
On peut aussi ajouter une formule mathématique centrées hors du texte entre deux symboles \textbf{\$\$}. Exemple:

\begin{columns}
  \begin{column}{0.45\textwidth}
    $|x|$ is positive for any value of $x$,
    we can define it like so
    $$x =
    \begin{cases}
      -x & \text{si }x < 0\\
      x & \text{sinon}.
    \end{cases}$$

    Be aware that
    $$|x + y| \neq |x| + |y|.$$
    However, we have the triangle inequality
    $$|x + y| \leq |x| + |y|$$
    for any $x,y \in \mathbb{C}$.
  \end{column}
  \begin{column}{0.45\textwidth}
    \begin{lstlisting}
\usepackage{amsmath} % for \begin{cases}
\usepackage{amssymb} % for \mathbb
...
$|x|$ is positive for
any value of $x$,
we can define it like so
$$x =
\begin{cases}
  -x & \text{si }x < 0\\
  x & \text{sinon}.
\end{cases}$$

Be aware that
$$|x + y| \neq |x| + |y|.$$
However, we have the triangle inequality
$$|x + y| \leq |x| + |y|$$
for any $x,y \in \mathbb{C}$.
    \end{lstlisting}
  \end{column}
\end{columns}
\end{frame}


\begin{frame}[fragile]{L'environnement mathématique}
\framesubtitle{Formules numérotées}

Un environnement équation est prévu pour des formules plus longues, elles seront automatiquement centrées et numérotées pour être référencées

\begin{columns}
  \begin{column}{0.6\textwidth}
I like trains and the equation~\eqref{eq:euler}
\begin{align}
  \label{eq:euler}
  e^{i\pi} + 1 & = 0\\
  \notag
  p(x) & = \frac{1}{\sigma \sqrt{2\pi}}
  \exp
    \left(-\frac{(x-\mu)^2}
                {2\sigma^2}\right).
\end{align}
I also know that
\begin{align*}
  1 + 1 & = 2 & 1 + 2 & = 3\\
  2 + 3 & = 5 & 3 + 5 & = 8.
\end{align*}
  \end{column}
  \begin{column}{0.4\textwidth}
    \begin{lstlisting}
\usepackage{amsmath} % for eqref
...
I like trains and
the equation~\eqref{eq:euler}
\begin{align}
  \label{eq:euler}
  e^{i\pi} + 1 & = 0\\
  \notag
  p(x) & = \frac{1}{\sigma \sqrt{2\pi}}
  \exp
    \left(-\frac{(x-\mu)^2}
    {2\sigma^2}\right).
\end{align}
I also know that
\begin{align*}
  1 + 1 & = 2 & 1 + 2 & = 3\\
  2 + 3 & = 5 & 3 + 5 & = 8.
\end{align*}
    \end{lstlisting}
  \end{column}
\end{columns}

\end{frame}

\begin{frame}[fragile]{L'environnement mathématique}
\framesubtitle{Variable à plusieurs lettres}

Attention aux yeux du lecteurs (surtout ceux ayant un compas à portée de main).
$cube = c \cdot u \cdot b \cdot e = c \times u \times b \times e$.
Les variables plusieurs lettres doivent être différenciées de celles à une seule lettre.

%\begin{columns}
%  \begin{column}{0.6\textwidth}
\begin{center}
  \begin{tabular}{|c|c|}
    \hline
    Bad & Good\\
    \hline
    $cube(x) = x^3$ & $\mathrm{cube}(x) = x^3$\\
    \hline
    $flux_{in}(k_{orig}) = flux_{out}(k_{dest})$ & $\mathrm{flux}_{\text{in}}(k_{\text{orig}}) = \mathrm{flux}_{\text{out}}(k_{\text{dest}})$\\
    \hline
  \end{tabular}
\end{center}
%  \end{column}
%  \begin{column}{0.4\textwidth}
\begin{lstlisting}
\begin{center}
  \begin{tabular}{|c|c|}
    \hline
    Bad & Good\\
    \hline
    $cube(x) = x^3$ & $\mathsf{cube}(x) = x^3$\\
    \hline
    $flux_{in}(k_{orig}) = flux_{out}(k_{dest})$ & $\mathsf{flux}_{\text{in}}(k_{\text{orig}}) = \mathsf{flux}_{\text{out}}(k_{\text{dest}})$\\
    \hline
  \end{tabular}
\end{center}
\end{lstlisting}
%  \end{column}
%\end{columns}
\textbf{Problème} Le code se ralonge (solution slide~\ref{noexcuse}).
\end{frame}

\begin{frame}[fragile]{L'environnement mathématique}
\framesubtitle{Les classes}
Les espaces du code sont ignorés en math mode.
Comment \TeX détermine l'espacement à faire ?

Il distingue 8 classes.
Chaque symbole, caractère ou sous-formule est dans une classe qui détermine l'espacement autour de lui.
\begin{center}
  \begin{tabular}{cp{0.5\textwidth}l}
    Ordinary & \lstinline|/|, sous-formule (en général) & \lstinline|\mathord| or \lstinline|{}|\\
    Large operator & \lstinline|\sum|, \lstinline|\prod| & \lstinline|\mathop|\\
    Binary operation & \lstinline|+| & \lstinline|\mathbin|\\
    Relation & \lstinline|=|, \lstinline|:| & \lstinline|\mathrel|\\
    Opening & \lstinline|(| & \lstinline|\mathopen|\\
    Closing & \lstinline|)| & \lstinline|\mathclose|\\
    Punctuation & \lstinline|,| & \lstinline|\mathpunct|\\
    Variable family & \lstinline|x| & \\
    Interne & aucun symbole seul. Sous-formule avec fraction ou \lstinline|\left..\right| & \lstinline|\mathinner|\\
  \end{tabular}
\end{center}
\end{frame}


\begin{frame}[fragile]{L'environnement mathématique}
\framesubtitle{Large Operators}
Ces opérateurs mathématiques sont $\lim, \min, \max, \sum, \prod, \ldots$.
Quelle différence ? Leurs indices et exposant sont au dessus et en dessous et pas à leur droite.

\begin{columns}
  \begin{column}{0.6\textwidth}
\begin{align*}
  \min_{x \in \mathbb{R}^n} \|x\|\\
  \sum_{i = 1}^n x_i & = 1
\end{align*}

$\min_{x \in \mathbb{R}^n} \|x\|$ tel que $\sum_{i = 1}^n x_i = 1$.

$\min\limits_{x \in \mathbb{R}^n} \|x\|$ tel que $\sum\limits_{i = 1}^n x_i = 1$.
  \end{column}
  \begin{column}{0.4\textwidth}
\begin{lstlisting}
\begin{align*}
  \min_{x \in \mathbb{R}^n} \|x\|\\
  \sum_{i = 1}^n x_i & = 1
\end{align*}

$\min_{x \in \mathbb{R}^n} \|x\|$ tel que $\sum_{i = 1}^n x_i = 1$.

$\min\limits_{x \in \mathbb{R}^n} \|x\|$ tel que $\sum\limits_{i = 1}^n x_i = 1$.
\end{lstlisting}
  \end{column}
\end{columns}
\end{frame}

\begin{frame}[fragile]{L'environnement mathématique}
\framesubtitle{Binary Operations and Relations}
Tableau pris de ``Handbook of Writing for the Mathematical Sciences'', Nicholas J. Higham.
\begin{tabular}{cc|cc}
  Relation or Binary operation & Exemple & Ordinary symbol & Exemple\\
  \hline
  \lstinline|:| & $\{\,z : |z| \leq 1\,\}$ & \lstinline|\colon| & $f \colon A \to B$\\
  \lstinline|\mid| & $\{\,x \mid x > 0\,\}$ & \lstinline|\vert| ou \lstinline$|$ & $|z|$\\
  \lstinline|\setminus| & $\mathbb{R} \setminus \{0\}$ & \lstinline|\backslash| & $p \backslash n$\\
  \lstinline|\parallel| & $\vec{u} \parallel \vec{v}$ & \lstinline|\Vert| ou \lstinline$\|$ & $\|A\|$\\
  \lstinline|\perp| & $\vec{u} \perp \vec{v}$ & \lstinline|\bot| & $x_\bot$\\
  \lstinline|\in| & $x \in \mathbb{R}$ & $\epsilon$ & $\epsilon > 0$\\
\end{tabular}

\end{frame}


\begin{frame}[fragile]{L'environnement mathématique}
\framesubtitle{Définition de commandes, plus d'excuse !}
\label{noexcuse}
\begin{lstlisting}
\newcommand{\fin}{\mathsf{flux}_{\text{in}}}
\newcommand{\fout}{\mathsf{flux}_{\text{out}}}
% if \kor already exists
\renewcommand{\kor}{k_{\text{orig}}}
\newcommand{\kde}{k_{\text{dest}}}
\DeclareMathOperator{\pot}{potato} % mieux que \newcommand{\mathop{\mathrm{..}}}
% \min already exists: Trick for ``reDeclareMathOperator''
\let\min\relax% Set equal to \relax so that LaTeX thinks it's not defined
\DeclareMathOperator{\min}{minimum}
\newcommand{\badet}{et}
\newcommand{\goodet}{\mathbin{\mathrm{et}}}
\end{lstlisting}
\begin{columns}
  \begin{column}{0.45\textwidth}
    \begin{lstlisting}
\[ \alpha >> \beta \badet <x,y> = 0 => \]
    \end{lstlisting}
  \end{column}
  \begin{column}{0.35\textwidth}
    \[ \alpha >> \beta \badet <x,y> = 0 => \]
  \end{column}
  \begin{column}{0.2\textwidth}
    \includegraphics[height=3em]{i_dont_want_to_live_on_this_planet_anymore.jpg}
  \end{column}
\end{columns}
\begin{columns}
  \begin{column}{0.45\textwidth}
    \begin{lstlisting}
\[ \alpha \gg \beta \goodet \langle x,y \rangle = 0 \Rightarrow \]
    \end{lstlisting}
  \end{column}
  \begin{column}{0.35\textwidth}
    \[ \alpha \gg \beta \goodet \langle x,y \rangle = 0 \Rightarrow \]
  \end{column}
  \begin{column}{0.2\textwidth}
    \includegraphics[height=3em]{success_kid.jpg}
  \end{column}
\end{columns}
\end{frame}

\begin{frame}[fragile]{L'environnement mathématique}
\framesubtitle{Forcer un espacement}
Rarement utile !

\begin{center}
  \begin{tabular}{ll}
    Commande & espacements en mu (espace normal${}={}$6mu)\\
    \hline
    \lstinline|\!|     & $-$3\\
    \lstinline|\,|     &  3\\
    \lstinline|\:|     &  4\\
    \lstinline|\;|     &  5\\
    \lstinline|\ |     &  6\\
    \lstinline|\quad|  & 18\\
    \lstinline|\qquad| & 36
  \end{tabular}
\end{center}
\end{frame}

\begin{frame}[fragile]{L'environnement mathématique}
\framesubtitle{Forcer un espacement : Exemples}
\begin{columns}
  \begin{column}{0.5\textwidth}
    \begin{lstlisting}
\begin{align*}
  a & = u + v + w + x + y\\
    & \quad + z
\end{align*}
    \end{lstlisting}
  \end{column}
  \begin{column}{0.5\textwidth}
\begin{align*}
  a & = u + v + w + x + y\\
    & \quad + z
\end{align*}
  \end{column}
\end{columns}

Erreur courante: les ensembles besoin d'espacement (i.e. \lstinline|\,|) en compréhension mais pas en extension.
\begin{columns}
  \begin{column}{0.5\textwidth}
    \begin{lstlisting}
\begin{align*}
  \mathbb{R}_+ & = \{\, x \in \mathbb{R} \mid R \geq 0 \,\}\\
  \mathbb{R}_+ & = \{\, x \in \mathbb{R} : R \geq 0 \,\}\\
  \mathbb{N} & = \{0, 1, 2, 3, 4, \ldots\}\\
\end{align*}
    \end{lstlisting}
  \end{column}
  \begin{column}{0.5\textwidth}
\begin{align*}
  \mathbb{R}_+ & = \{\, x \in \mathbb{R} \mid R \geq 0 \,\}\\
  \mathbb{R}_+ & = \{\, x \in \mathbb{R} : R \geq 0 \,\}\\
  \mathbb{N} & = \{0, 1, 2, 3, 4, \ldots\}\\
\end{align*}
  \end{column}
\end{columns}
\end{frame}

\subsection{La physique}
\begin{frame}[fragile]
  \frametitle{Les unités}
  \lstinline|\usepackage{siunitx}|
  \begin{center}
    \begin{tabular}{ll}
      \num{314e-2} & \lstinline|\num{314e-2}|\\
      \ang{42} & \lstinline|\ang{42}|\\
      \si{g_{polymer}~mol_{cat}.s^{-1}} &
      \lstinline|\si{g_{polymer}~mol_{cat}.s^{-1}}|\\
      \si{\square\volt\cubic\lumen\per\farad} &
      \lstinline|\si{\square\volt\cubic\lumen\per\farad}|\\
      \SI{e-6}{\meter\per\second\per\ohm} &
      \lstinline|\SI{e-6}{\meter\per\second\per\ohm}|\\
      \SI[per-mode=symbol]{5.3e9}{m\per s} &
      \lstinline|\SI[per-mode=symbol]{5.3e9}{m\per s}|\\
      \SI[per-mode=symbol]{5.3e9}{\meter\per\second\per\ohm} &
      \lstinline|\SI[per-mode=symbol]{5.3e9}{\meter\per\second\per\ohm}|\\
      \SI[per-mode=fraction]{5e6}{\joule\per\second} &
      \lstinline|\SI[per-mode=fraction]{5e6}{\joule\per\second}|\\
      \SI{-273.15}{\celsius} &
      \lstinline|\SI{-273.15}{\celsius}|
    \end{tabular}
  \end{center}
  Super doc sur \url{http://ctan.org/pkg/siunitx}
\end{frame}

\subsection{La chimie}
\begin{frame}[fragile]
  \centering
  \frametitle{La chimie}
  \begin{lstlisting}
\usepackage{chemfig}
...
\chemfig{*6(-=(-CH_2OH)-(-COOH)=-=)}
  \end{lstlisting}
  \begin{center}
    \chemfig{*6(-=(-CH_2OH)-(-COOH)=-=)}
  \end{center}
  \begin{lstlisting}
\usepackage[version=3]{mhchem}
...
$$\ce{3H2O + 1/2H2O -> AgCl2- + H2_{(aq)}}$$
  \end{lstlisting}
  $$\ce{3H2O + 1/2H2O -> AgCl2- + H2_{(aq)}}$$
\end{frame}

\subsection{Les circuits}
\begin{frame}[fragile]
  \frametitle{Les circuits}
  \begin{lstlisting}
    \usepackage{circuitikz}
    ...
		\shorthandoff{:!} % Pour certaines versions de circuitikz
    \begin{circuitikz}
      \draw (0,0) to [sI, v=$V_2$] (0,-3);
      \draw (6,-3) to[short, i = $I_2$] (0,-3);
      \draw (0,0) to [R = R, v = $V_R$] (3,0);
      \draw (3,0) to [L = L, v = $V_L$] (6,0);
      \draw (6,0) to [C = C, v = $V_C$] (6,-3);
    \end{circuitikz}
		\shorthandon{:!} % Pour certaines versions de circuitikz
  \end{lstlisting}
  \begin{center}
		\shorthandoff{:!}
    \begin{circuitikz}
      \draw (0,0) to [sI, v=$V_2$] (0,-3);
      \draw (6,-3) to[short, i = $I_2$] (0,-3);
      \draw (0,0) to [R = R, v = $V_R$] (3,0);
      \draw (3,0) to [L = L, v = $V_L$] (6,0);
      \draw (6,0) to [C = C, v = $V_C$] (6,-3);
    \end{circuitikz}
		\shorthandon{:!}
  \end{center}
\end{frame}

\subsection{Inclure du code}
\begin{frame}[fragile]
  \frametitle{Inclure du code}
  \begin{lstlisting}[mathescape=true]
\begin{lstlisting}
if a == b:
  return 0
else:
  return 1
\$$end{lstlisting}
  \end{lstlisting}
donne
  \begin{lstlisting}[language=Python]
if a == b:
  return 0
else:
  return 1
  \end{lstlisting}

  Il y a aussi
  \begin{lstlisting}
\lstinputlisting[caption={...},label=...]{main.py}
  \end{lstlisting}
  et
  \begin{lstlisting}
\lstinline|if a == b|
  \end{lstlisting}
  qui donne \lstinline[language=Python]|if a == b|.
\end{frame}

%%%%%%%%%%%%%

\AtBeginSection[]{} %% stop TOC

\section{Exercices}
\begin{frame}
\frametitle{Exerçons-nous}
	\begin{itemize}
		\item<1-> Télécharger le document \textbf{exemple.pdf}
		\item<2-> Reproduire une structure similaire :
			\begin{itemize}
				\item page de titre
				\item table des matières
				\item liste, tableau, figure
				\item math en ligne, hors-ligne
				\item références
				\item \dots
			\end{itemize}
		\item<3-> Chercher de l'information :
			\begin{itemize}
				\item \url{http://en.wikibooks.org/wiki/LaTeX}
				\item \url{http://www.grappa.univ-lille3.fr/FAQ-LaTeX}
                \item \url{http://www.andy-roberts.net/writing/latex}
                \item \url{http://ctan.org/pkg/packagename} ou \lstinline[language=sh,morekeywords={texdoc}]{$ texdoc packagename}
				\item Google est ton ami !
                \item \url{http://www.sharelatex.com/learn}
                \item La version de StackExchange spécialisée pour le \TeX:
      \url{tex.stackexchange.com}.
				\item Livres:
				\begin{itemize}
					\item \LaTeX HowTo par Sébastien Combéfis (EN/FR)
					\item Framabook \LaTeX
				\end{itemize}
				\item \url{http://www.tablesgenerator.com/}
			\end{itemize}
	\end{itemize}
\end{frame}


\end{document}
